\documentclass{report}

\usepackage[a4paper, margin=0.5in]{geometry}
\usepackage{graphicx}
\usepackage{hyperref}
\usepackage{xurl}
\usepackage{amsmath}

\begin{document}

\begin{titlepage}
    \centering

    \includegraphics[width=8cm]{images/nfl_logo.png}\par\vspace{1cm}

    \vspace{0.5cm}
    {\huge\bfseries A Summer Project in NFL Analytics \par}
    \vspace{2cm}

    {\Large Adam Lynch \par}
    \vspace{1cm}

    {\large May 13, 2025 - May 13, 2025 \par}

    \vfill
\end{titlepage}

\tableofcontents
\newpage

\chapter{Rating Systems}

\section{Simple Rating System (SRS)}

The Simple Rating System (SRS) is a straightforward method used to evaluate teams based on their average point differential adjusted for the strength of their opponents. 
Though its exact origins are unclear in our research, Doug Drinen of Sports Reference discussed the SRS and its calculation in a series of blog posts in 2006 \cite{drinen06-1,drinen06-2}.
Drinen provides an excellent overview of the SRS which we recommend reading.
We present Drinen's \textit{transitivity} formulation of the SRS here, which will be useful as a base to build upon later.

The argument of transitivity is that if team A is better than team B, and team B is better than team C, then team A is better than team C.
In fact, in a perfect world, team A beats team B by $X$ points, team B beats team C by $Y$ points, and team A beats team C by $X+Y$ points.
Following this argument, if we define the SRS as a teams point differential in a game against and arbitrarily average opponent, we can set up the following system of equations:

\begin{equation}
    \begin{aligned}
        P_{1,A} - P_{1,B} &= SRS_A - SRS_B \\
        P_{2,B} - P_{2,C} &= SRS_B - SRS_C \\
        P_{3,A} - P_{3,C} &= SRS_A - SRS_C
    \end{aligned}
\end{equation}

Here we use the notation $P_{i,j}$ to represent the points scored by team $j$ in game $i$.
Unfortunately, this system might not be solvable -- especially so if two teams play multiple games against each other.
The more appropriate system instead included an error term $\epsilon$, where the SRS ratings are such that the sum of square of all $\epsilon$ is minimized (i.e. $\epsilon_1^2+\epsilon_2^2+\epsilon_3^2$).

\begin{equation}
    \begin{aligned}
        P_{1,A} - P_{1,B} &= SRS_A - SRS_B + \epsilon_1 \\
        P_{2,B} - P_{2,C} &= SRS_B - SRS_C + \epsilon_2 \\
        P_{3,A} - P_{3,C} &= SRS_A - SRS_C + \epsilon_3
    \end{aligned}
\end{equation}



\section{Predictive SRS (PSRS)}

\section{Predictive SRS with Exogenous Regressors (PSRSX)}

\begin{thebibliography}{9}
    \bibitem{drinen06-1} Drinen D. (2006), ``A very simple ranking system'', \textit{Pro-Football-Reference}, \url{https://web.archive.org/web/20161031224357/http://www.pro-football-reference.com/blog/index4837.html}
    \bibitem{drinen06-2} Drinen D. (2006), ``Another ranking system'', \textit{Pro-Football-Reference}, \url{https://web.archive.org/web/20161102124021/http://www.pro-football-reference.com/blog/indexba52.html?p=39}
\end{thebibliography}

\end{document}
